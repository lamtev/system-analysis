\documentclass[a4paper,14pt]{extarticle}

\usepackage[utf8x]{inputenc}
\usepackage[T1,T2A]{fontenc}
\usepackage[russian]{babel}
\usepackage{hyperref}
\usepackage{indentfirst}
\usepackage{here}
\usepackage{array}
\usepackage[table]{xcolor}
\usepackage{datetime}
\usepackage{multirow}
\usepackage{hhline}
\usepackage{graphicx}
\usepackage{caption}
\usepackage{subcaption}
\usepackage{chngcntr}
\usepackage{amsmath}
\usepackage{amssymb}
\usepackage{pgfplots}
\usepackage{pgfplotstable}
\usepackage[left=2cm,right=2cm,top=2cm,bottom=2cm,bindingoffset=0cm]{geometry}
\usepackage{multicol}
\usepackage{askmaps}
\usepackage{tikz}

\newcommand*\circled[1]{\tikz[baseline=(char.base)]{
            \node[shape=circle,draw,inner sep=2pt] (char) {#1};}}

\renewcommand{\not}[1]{\mkern 1.5mu\overline{\mkern-1.5mu#1\mkern-1.5mu}\mkern 1.5mu}
\renewcommand{\le}{\ensuremath{\leqslant}}
\renewcommand{\leq}{\ensuremath{\leqslant}}
\renewcommand{\ge}{\ensuremath{\geqslant}}
\renewcommand{\geq}{\ensuremath{\geqslant}}
\renewcommand{\epsilon}{\ensuremath{\varepsilon}}
\renewcommand{\phi}{\ensuremath{\varphi}}

\counterwithin{figure}{section}
\counterwithin{equation}{section}
\counterwithin{table}{section}
\newcommand{\sign}[1][5cm]{\makebox[#1]{\hrulefill}} % Поля подписи и даты
\graphicspath{{pics/}} % Путь до папки с картинками
\captionsetup{justification=centering,margin=1cm}
\def\arraystretch{1.3}

\begin{document}

\begin{titlepage}
\begin{center}
	Санкт-Петербургский политехнический университет Петра Великого\\[0.3cm]
	Институт компьютерных наук и технологий \\[0.3cm]
	Кафедра компьютерных систем и программных технологий\\[4cm]
	
	\textbf{Расчётное задание №6}\\[2mm]
	\textbf{Дисциплина:} Системный анализ и принятие решений\\[2mm]
	\textbf{Тема:} Дискретное программирование. Задача коммивояжёра\\[2mm]
	Вариант 39\\[6.5cm]
\end{center}

\begin{flushleft}
	\hspace*{5mm} Выполнил студент гр. 33501/4  \hspace*{3cm}\sign[3cm]\hspace*{2mm} А.Ю. Ламтев\\
	\hspace*{10.85cm} (подпись)\\[2.5mm]
	\hspace*{5mm} Преподаватель \hspace*{6.45cm}\sign[3cm]\hspace*{2mm} С.С. Сабонис\\
	\hspace*{10.85cm} (подпись)\\[2.5mm]
	\hspace*{11.1cm} <<\underline{\the\day}>> \underline{\hspace{5mm}ноября\hspace{5mm}} \the\year\hspace{1mm} г.
\end{flushleft}

\vfill

\begin{center}
	Санкт-Петербург\\
	\the\year
\end{center}
\end{titlepage}
\addtocounter{page}{1}

\section{Задание}

Дана задача нелинейного программирования:

\begin{displaymath}
	max \left( -17 x^2_1 - 23 x^2_2 + 8 x_1 x_2 + 182 x_1 + 266 x_2 \right)
\end{displaymath}

\begin{enumerate}

	\item Записать условие оптимальности и решить задачу аналитически.
	
	\item Решить задачу методом релаксации.
	
	\item Решить задачу методом наискорейшего подъема.
	
	\item Решить задачу методом Ньютона.	
	
	\item Решить задачу методом сопряжённых градиентов.

	\item Решить задачу методом Бройдена.
	
	\item Решить задачу методами 2--6, выбрав ещё 3 начальных точки. Графики сгруппировать по методам: на одном графике траектории поиска решения задачи из четырёх разных начальных точек одним методом, на другом графике – другим методом и т.д. 

\end{enumerate}

\section{Решение}

\subsection{Аналитическое решение}

Решим задачу аналитически. Для этого запишем условие оптимальности:

\begin{equation}
\label{eq:opt-cond}
	\nabla f(X) \Big|_{X = X^*} = \left(  \frac{\partial f}{\partial x_1}(X) \hspace{7mm} \frac{\partial f}{\partial x_2}(X) \right)^T \Big|_{X = X^*} = 0, \hspace{5mm} X = 
	%
	\begin{pmatrix}
		x_1
		\\
		x_2
	\end{pmatrix}
\end{equation}

Вычислим частные производные первого порядка целевой функции:

\begin{equation*}
	\frac{\partial f}{\partial x_1}(X) = -34 x_1 + 8 x_2 + 182
\end{equation*}

\begin{equation*}
	\frac{\partial f}{\partial x_2}(X) = -46 x_2 + 8 x_1 + 266
\end{equation*}

Подставим их в уравнение \ref{eq:opt-cond} и решим полученную систему уравнений:

\begin{equation*}
	\begin{pmatrix}
		-34 x_1 + 8 x_2 + 182
		\\
		-46 x_2 + 8 x_1 + 266
	\end{pmatrix}
	%
	= 0
	%
	\Rightarrow
	\begin{pmatrix}
		-34 & 8
		\\
		-46 & 8 x_1
	\end{pmatrix}
	\cdot X^* =
	\begin{pmatrix}
		-182
		\\
		-266
	\end{pmatrix}
	\Rightarrow
\end{equation*}

\begin{equation*}
	\Rightarrow X^* = 
	\begin{pmatrix}
		7
		\\
		7
	\end{pmatrix}
	\text{ -- оптимальная точка}
\end{equation*}

Запишем достаточное условие максимума:

\begin{equation}
\label{eq:enough-cond}
	H(X) \Big|_{X=X^*} = 
	\begin{pmatrix}
		\frac{\partial^2 f}{\partial x^2_1}(X) && \frac{\partial^2 f}{\partial x_1 \partial x_2}(X)
		\\
		\frac{\partial^2 f}{\partial x_1 \partial x_2}(X) && \frac{\partial^2 f}{\partial x^2_2}(X)
	\end{pmatrix}
	\Big|_{X=X^*}
	\text{-- отрицательно определена}
\end{equation}

Вычислим частные производные второго порядка целевой функции:

\begin{equation*}
	\frac{\partial^2 f}{\partial x^2_1}(X) = -34
\end{equation*}

\begin{equation*}
	\frac{\partial^2 f}{\partial x_1 \partial x_2}(X) = 8
\end{equation*}

\begin{equation*}
	\frac{\partial^2 f}{\partial x^2_2}(X) = - 46
\end{equation*}

Подставим их в уравнение \ref{eq:enough-cond} и получим:

\begin{equation*}
	H(X) = 
	\begin{pmatrix}
		-34 && 8
		\\
		8 && -46
	\end{pmatrix}
\end{equation*}

Согласно критерию Сильвестра (Действительная квадратичная форма является отрицательно определенной тогда и только тогда, когда знаки главных миноров ее матрицы чередуются, причем $\Delta_1 < 0$), полученная матрица Гёссе отрицательно определена, т.к $\Delta_1 = h_{11} = -34 < 0$ и $\Delta_2 =%
\begin{vmatrix}% 
	h_{11} & h_{12}%
	\\
	h_{21} & h_{22}
\end{vmatrix} =%
=
\begin{vmatrix}% 
	-34 & 8%
	\\
	8 & -46
\end{vmatrix} =1500 > 0$

Следовательно $X = \begin{pmatrix}
	7
	\\
	7
\end{pmatrix}$ -- решение.

\end{document}