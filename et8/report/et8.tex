\documentclass[a4paper,14pt]{extarticle}

\usepackage[utf8x]{inputenc}
\usepackage[T1,T2A]{fontenc}
\usepackage[russian]{babel}
\usepackage{hyperref}
\usepackage{indentfirst}
\usepackage{here}
\usepackage{array}
\usepackage[table]{xcolor}
\usepackage{datetime}
\usepackage{multirow}
\usepackage{hhline}
\usepackage{mathtools,cancel}
\usepackage{forest}
\usepackage{graphicx}
\usepackage{caption}
\usepackage{subcaption}
\usepackage{chngcntr}
\usepackage{amsmath}
\usepackage{amssymb}
\usepackage{pgfplots}
\usepackage{pgfplotstable}
\usepackage[left=2cm,right=2cm,top=2cm,bottom=2cm,bindingoffset=0cm]{geometry}
\usepackage{multicol}
\usepackage{askmaps}
\usepackage{tikz}

\newcommand*\circled[1]{\tikz[baseline=(char.base)]{
            \node[shape=circle,draw,inner sep=2pt] (char) {#1};}}

\DeclareMathOperator*{\argmin}{argmin}

\renewcommand{\not}[1]{\mkern 1.5mu\overline{\mkern-1.5mu#1\mkern-1.5mu}\mkern 1.5mu}
\renewcommand{\le}{\ensuremath{\leqslant}}
\renewcommand{\leq}{\ensuremath{\leqslant}}
\renewcommand{\ge}{\ensuremath{\geqslant}}
\renewcommand{\geq}{\ensuremath{\geqslant}}
\renewcommand{\epsilon}{\ensuremath{\varepsilon}}
\renewcommand{\phi}{\ensuremath{\varphi}}

\counterwithin{figure}{section}
\counterwithin{equation}{section}
\counterwithin{table}{section}
\newcommand{\sign}[1][5cm]{\makebox[#1]{\hrulefill}} % Поля подписи и даты
\graphicspath{{pics/}} % Путь до папки с картинками
\captionsetup{justification=centering,margin=1cm}
\def\arraystretch{1.3}

\begin{document}

\begin{titlepage}
\begin{center}
	Санкт-Петербургский политехнический университет Петра Великого\\[0.3cm]
	Институт компьютерных наук и технологий \\[0.3cm]
	Кафедра компьютерных систем и программных технологий\\[4cm]
	
	\textbf{Расчётное задание №6}\\[2mm]
	\textbf{Дисциплина:} Системный анализ и принятие решений\\[2mm]
	\textbf{Тема:} Дискретное программирование. Задача коммивояжёра\\[2mm]
	Вариант 39\\[6.5cm]
\end{center}

\begin{flushleft}
	\hspace*{5mm} Выполнил студент гр. 33501/4  \hspace*{3cm}\sign[3cm]\hspace*{2mm} А.Ю. Ламтев\\
	\hspace*{10.85cm} (подпись)\\[2.5mm]
	\hspace*{5mm} Преподаватель \hspace*{6.45cm}\sign[3cm]\hspace*{2mm} С.С. Сабонис\\
	\hspace*{10.85cm} (подпись)\\[2.5mm]
	\hspace*{11.1cm} <<\underline{\the\day}>> \underline{\hspace{5mm}ноября\hspace{5mm}} \the\year\hspace{1mm} г.
\end{flushleft}

\vfill

\begin{center}
	Санкт-Петербург\\
	\the\year
\end{center}
\end{titlepage}
\addtocounter{page}{1}

\section{Задание}

\textbf{Задача АС9}\\

Компания решила довести число своих машин до $N$. Президент компании интересуется, как будут справляться с ремонтом имеющиеся $R$ механиков. Средняя скорость прибытия на ремонт равна $K$ $\frac{\text{раз}}{\text{час}}$ для каждой машины, средняя скорость обслуживания – $M$  $\frac{\text{машин}}{\text{час}}$. Если число ожидающих ремонта машин становится больше, чем число ремонтируемых машин, то механики отказываются от перекуров и увеличивают скорость обслуживания на $25$\%.\\

Определить:
\begin{itemize}
	\item вероятность того, что все машины работают и механики отдыхают
	\item среднее число ожидающих ремонта машин
	\item среднее число машин в системе (машины в очереди и на обслуживании)
	\item среднее время ожидания начала ремонта
	\item среднее время нахождения в системе (ожидание и ремонт)
\end{itemize}

Исходные данные:\\

\begin{equation*}
	N = 9,\ \ K = 0.05,\ \ M = 0.5,\ \ R = 2
\end{equation*}

\section{Решение}

В постановке задачи описана СМО M/M/2//9\\ с $\lambda = 0.05$ и $\mu = \begin{cases} \mu_0 = 0.5,\ \ \text{заявок в очереди} \leq \text{заявок на обслуживании}\\ 1.25 \mu_0,\ \  \text{иначе}\end{cases}$. %На рис. \ref{pic:smo-model} представлена модель такой СМО.

%\begin{figure}[H]
%\begin{center}
%	\includegraphics[width=1\textwidth]{smo-model}
%	\caption{Модель СМО}
%	\label{pic:smo-model}
%\end{center}
%\end{figure}

Построим граф гибели-размножения, он изображен на рис. \ref{pic:smo-graph}.

\begin{figure}[H]
\begin{center}
\begin{tikzpicture}[->,>=stealth',shorten >=1pt,auto,node distance=1.75cm,
                    thick,main node/.style={circle,draw,font=\sffamily\Large\bfseries}]

  \node[main node] (0) {0};
  \node[main node] (1) [right of = 0]{1};
  \node[main node] (2) [right of = 1] {2};
  \node[main node] (3) [right of = 2] {3};
  \node[main node] (4) [right of = 3] {4};
  \node[main node] (5) [right of = 4] {5};
  \node[main node] (6) [right of = 5] {6};
  \node[main node] (7) [right of = 6] {7};
  \node[main node] (8) [right of = 7] {8};
  \node[main node] (9) [right of = 8] {9};

  \path[every node/.style={font=\sffamily\small}]
    (0) edge [bend left] node[above] {$9\lambda$} (1)
    (1) edge [bend left] node[above] {$8\lambda$} (2)
    (2) edge [bend left] node[above] {$7\lambda$} (3)
    (3) edge [bend left] node[above] {$6\lambda$} (4)
    (4) edge [bend left] node[above] {$5\lambda$} (5)
    (5) edge [bend left] node[above] {$4\lambda$} (6)
    (6) edge [bend left] node[above] {$3\lambda$} (7)
    (7) edge [bend left] node[above] {$2\lambda$} (8)
    (8) edge [bend left] node[above] {$\lambda$} (9)
    
    (9) edge [bend left]  node[below]   {$2\mu_1$} (8)
    (8) edge [bend left] node[below] {$2\mu_1$} (7)
    (7) edge [bend left] node[below] {$2\mu_0$} (6)
    (6) edge [bend left] node[below] {$2\mu_0$} (5)
    (5) edge [bend left] node[below] {$2\mu_0$} (4)
    (4) edge [bend left] node[below] {$2\mu_0$} (3)
    (3) edge [bend left] node[below] {$2\mu_0$} (2)
    (2) edge [bend left] node[below] {$2\mu_0$} (1)
    (1) edge [bend left] node[below] {$\mu_0$} (0);
  
\end{tikzpicture}
\caption{Граф гибели-размножения СМО}
\label{pic:smo-graph}
\end{center}
\end{figure}

Составим и решим систему уравнений \ref{eq:1}.

\begin{equation}
\label{eq:1}
\begin{cases}
	P_0 \cdot 9 \lambda = P_1 \cdot \mu_0 \\
	P_1 \cdot 8 \lambda = P_2 \cdot 2 \mu_0 \\
	P_2 \cdot 7 \lambda = P_3 \cdot 2 \mu_0 \\
	P_3 \cdot 6 \lambda = P_4 \cdot 2 \mu_0 \\
	P_4 \cdot 5 \lambda = P_5 \cdot 2 \mu_0 \\
	P_5 \cdot 4 \lambda = P_6 \cdot 2 \mu_0 \\
	P_6 \cdot 3 \lambda = P_7 \cdot 2 \mu_0 \\
	P_7 \cdot 2 \lambda = P_8 \cdot 2 \mu_1 \\
	P_8 \cdot 1 \lambda = P_9 \cdot 2 \mu_1 \\
	\sum P_i = 1
\end{cases}
\begin{cases}
	P_1 = P_0 \cdot \frac{9 \lambda}{\mu_0} \\
	P_2 = P_0 \cdot \frac{36 \lambda^2}{\mu_0^2} \\
	P_3 = P_0 \cdot \frac{126 \lambda^3}{\mu_0^3} \\
	P_4 = P_0 \cdot \frac{378 \lambda^4}{\mu_0^4} \\
	P_5 = P_0 \cdot \frac{945 \lambda^5}{\mu_0^5} \\
	P_6 = P_0 \cdot \frac{1890 \lambda^6}{\mu_0^6} \\
	P_7 = P_0 \cdot \frac{2835 \lambda^7}{\mu_0^7} \\
	P_8 = P_0 \cdot \frac{2835 \lambda^7}{\mu_0^7 \mu_1} \\
	P_9 = P_0 \cdot \frac{2835 \lambda^7}{2 \mu_0^7 \mu_1^2} \\
	\sum P_i = 1
\end{cases}
\begin{cases}
	P_0 = 0.41060 \\
	P_1 = 0.36954 \\
	P_2 = 0.14782 \\
	P_3 = 0.05174 \\
	P_4 = 0.01552 \\
	P_5 = 0.00388 \\
	P_6 = 0.00078 \\
	P_7 = 0.00078 \\ 
	P_8 = 9.31246 \cdot 10^{-6} \\
	P_9 = 3.72498 \cdot 10^{-7} \\
\end{cases}
\end{equation}

Вычислим свойства системы:

\begin{equation*}
	\not{\lambda} = \sum_{i = 0}^9 \frac{9 - i}{9} \cdot \lambda \cdot P_i = 0.04497
\end{equation*}

\begin{equation*}
	\not{\mu} = \sum_{i = 0}^9 \mu_i \cdot P_i = 0.50042
\end{equation*}

\begin{equation*}
	\not{x} = \frac{1}{\not{\mu}} = 1.99829
\end{equation*}

\begin{equation*}
	\not{n_0} = \sum_{i = 3}^9 (i - 2) \cdot P_i = 0.10146
\end{equation*}

\begin{equation*}
	\not{j} = \sum_{i = 0}^9 i \cdot P_i = 0.91203
\end{equation*}

\begin{equation*}
	\not{K_\text{з}} = P_1 + 2 \cdot (P_2 + \dots + P_9) = 0.81057
\end{equation*}

\begin{equation*}
	\not{t_\text{ож}} = \frac{\not{j} - \not{K_\text{з}}}{\not{\lambda}(9 - \not{j})} = 0.27898
\end{equation*}

\begin{equation*}
	\not{t_\text{с}} = \not{t_\text{ож}} + \not{x} = 2.27727
\end{equation*}

\noindent Вероятность того, что все машины работают и механики отдыхают $P_0 = 0.41060$\\
Среднее число ожидающих ремонта машин $\not{n_0} = 0.10146$\\
Среднее число машин в системе $\not{j} = 0.91203$\\
Среднее время ожидания начала ремонта $\not{t_\text{ож}} = 0.27898$\\
Среднее время нахождения в системе $\not{t_\text{с}} = 2.27727$\\

\end{document}