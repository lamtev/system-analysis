\documentclass[a4paper,14pt]{extarticle}

\usepackage[utf8x]{inputenc}
\usepackage[T1,T2A]{fontenc}
\usepackage[russian]{babel}
\usepackage{hyperref}
\usepackage{indentfirst}
\usepackage{here}
\usepackage{array}
\usepackage[table]{xcolor}
\usepackage{datetime}
\usepackage{multirow}
\usepackage{hhline}
\usepackage{mathtools,cancel}
\usepackage{forest}
\usepackage{graphicx}
\usepackage{caption}
\usepackage{subcaption}
\usepackage{chngcntr}
\usepackage{amsmath}
\usepackage{amssymb}
\usepackage{pgfplots}
\usepackage{pgfplotstable}
\usepackage[left=2cm,right=2cm,top=2cm,bottom=2cm,bindingoffset=0cm]{geometry}
\usepackage{multicol}
\usepackage{askmaps}
\usepackage{tikz}

\newcommand*\circled[1]{\tikz[baseline=(char.base)]{
            \node[shape=circle,draw,inner sep=2pt] (char) {#1};}}

\DeclareMathOperator*{\argmin}{argmin}

\renewcommand{\not}[1]{\mkern 1.5mu\overline{\mkern-1.5mu#1\mkern-1.5mu}\mkern 1.5mu}
\renewcommand{\le}{\ensuremath{\leqslant}}
\renewcommand{\leq}{\ensuremath{\leqslant}}
\renewcommand{\ge}{\ensuremath{\geqslant}}
\renewcommand{\geq}{\ensuremath{\geqslant}}
\renewcommand{\epsilon}{\ensuremath{\varepsilon}}
\renewcommand{\phi}{\ensuremath{\varphi}}

\counterwithin{figure}{section}
\counterwithin{equation}{section}
\counterwithin{table}{section}
\newcommand{\sign}[1][5cm]{\makebox[#1]{\hrulefill}} % Поля подписи и даты
\graphicspath{{pics/}} % Путь до папки с картинками
\captionsetup{justification=centering,margin=1cm}
\def\arraystretch{1.3}

\begin{document}

\begin{titlepage}
\begin{center}
	Санкт-Петербургский политехнический университет Петра Великого\\[0.3cm]
	Институт компьютерных наук и технологий \\[0.3cm]
	Кафедра компьютерных систем и программных технологий\\[4cm]
	
	\textbf{Расчётное задание №6}\\[2mm]
	\textbf{Дисциплина:} Системный анализ и принятие решений\\[2mm]
	\textbf{Тема:} Дискретное программирование. Задача коммивояжёра\\[2mm]
	Вариант 39\\[6.5cm]
\end{center}

\begin{flushleft}
	\hspace*{5mm} Выполнил студент гр. 33501/4  \hspace*{3cm}\sign[3cm]\hspace*{2mm} А.Ю. Ламтев\\
	\hspace*{10.85cm} (подпись)\\[2.5mm]
	\hspace*{5mm} Преподаватель \hspace*{6.45cm}\sign[3cm]\hspace*{2mm} С.С. Сабонис\\
	\hspace*{10.85cm} (подпись)\\[2.5mm]
	\hspace*{11.1cm} <<\underline{\the\day}>> \underline{\hspace{5mm}ноября\hspace{5mm}} \the\year\hspace{1mm} г.
\end{flushleft}

\vfill

\begin{center}
	Санкт-Петербург\\
	\the\year
\end{center}
\end{titlepage}
\addtocounter{page}{1}

\section{Задание}

Задана замкнутая сеть массового обслуживания, включающая $M = 4$ узла. В сети циркулирует $N$ заявок в соответствии с матрицей передач, также заданы описания узлов как систем массового обслуживания (число каналов, интенсивность обслуживания). 

\begin{itemize}
	\item $N = 6$
	\item Матрица передач:
	\begin{equation*}
		\mathbb{P} = \begin{pmatrix}
			\sfrac{1}{9} & \sfrac{7}{18} & 0 & \sfrac{1}{2} \\
			\sfrac{7}{11} & 0 & 0 & \sfrac{4}{11} \\
			\sfrac{7}{12} & \sfrac{5}{12} & 0 & 0 \\
			0 & 0 & 1 & 0
		\end{pmatrix}
	\end{equation*}
	\item 1 узел: система $M/M/3$, $\mu = 9$
	\item 2 узел: система $M/M/3$, $\mu = 4$
	\item 3 узел: система $M/M/1$, $\mu = 9$
	\item 4 узел: система $M/M/2$, $\mu = 4$
\end{itemize}

Необходимо:
\begin{enumerate}
	\item Построить граф сети;
	\item Определить среднее число требований, среднее число ожидающих требований, среднее время пребывания и среднее время ожидания для каждого узла;
	\item Результаты оформить в итоговой таблице.
\end{enumerate}

\section{Построение графа сети}

На рис. \ref{pic:nmo-graph} изображён граф сети.

\begin{figure}[H]
\begin{center}
\begin{tikzpicture}[->,>=stealth',shorten >=2pt,auto,node distance=5cm,
                    thick,main node/.style={circle,draw,font=\sffamily\Large\bfseries}]

  \node[main node] (1){1};
  \node[main node] (2) [above right of = 1] {2};
  \node[main node] (3) [below right of = 2] {3};
  \node[main node] (4) [below left of = 3] {4};

  \path[every node/.style={font=\sffamily\small}]
    (1) edge [loop left]      node[above] {$1$} (1)
    (1) edge [bend left] node[above] {$\sfrac{7}{18}$} (2)
    (1) edge [bend right] node[below] {$\sfrac{1}{2}$} (4)
    (2) edge [bend left] node[below] {$\sfrac{7}{11}$} (1)
    (2) edge             node[pos=0.8] {$\sfrac{4}{11}$} (4)
    (3) edge [bend right] node[above] {$\sfrac{5}{12}$} (2)
    (3) edge             node[pos=0.8] {$\sfrac{7}{12}$} (1)
    (4) edge [bend right] node[below] {$1$} (3);
  
\end{tikzpicture}
\caption{Граф сети}
\label{pic:nmo-graph}
\end{center}
\end{figure}

\section{Расчет характеристик узлов сети}

В таблице \ref{tab:results} приведены результаты рассчитанных показателей для отдельных узлов сети.
\begin{table}[H]
	\begin{center}
		\caption{Результаты}
		\label{tab:results}
		\def\tabcolsep{4pt}
		\begin{tabular}{|c|c|c|c|c|}
			\hline
			& 1 узел & 2 узел & 3 узел & 4 узел \\
			\hline
			Ср. число требований $\not{n}$ & $0.89869  $ & $1.4815  $ & $1.41   $ & $2.2098 $ \\
			\hline
			Ср. число ожидающих требований $\not{l}$ & $0.013525 $ & $0.079674$ & $0.74088$ & $0.70422$ \\
			\hline
			Ср. время пребывания $\not{t_\text{с}}$ & $0.11281  $ & $0.26421 $ & $0.23414$ & $0.36694$ \\
			\hline
			Ср. время ожидания $\not{t_\text{ож}}$ & $0.0016977$ & $0.014209$ & $0.12302$ & $0.11694$ \\
			\hline	
		\end{tabular}
	\end{center}
\end{table}



\end{document}