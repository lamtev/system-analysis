\documentclass[a4paper,14pt]{extarticle}

\usepackage[utf8x]{inputenc}
\usepackage[T1,T2A]{fontenc}
\usepackage[russian]{babel}
\usepackage{hyperref}
\usepackage{indentfirst}
\usepackage{here}
\usepackage{array}
\usepackage[table]{xcolor}
\usepackage{datetime}
\usepackage{multirow}
\usepackage{hhline}
\usepackage{graphicx}
\usepackage{caption}
\usepackage{subcaption}
\usepackage{chngcntr}
\usepackage{amsmath}
\usepackage{amssymb}
\usepackage{pgfplots}
\usepackage{pgfplotstable}
\usepackage[left=2cm,right=2cm,top=2cm,bottom=2cm,bindingoffset=0cm]{geometry}
\usepackage{multicol}
\usepackage{askmaps}
\usepackage{tikz}

\newcommand*\circled[1]{\tikz[baseline=(char.base)]{
            \node[shape=circle,draw,inner sep=2pt] (char) {#1};}}

\renewcommand{\not}[1]{\mkern 1.5mu\overline{\mkern-1.5mu#1\mkern-1.5mu}\mkern 1.5mu}
\renewcommand{\le}{\ensuremath{\leqslant}}
\renewcommand{\leq}{\ensuremath{\leqslant}}
\renewcommand{\ge}{\ensuremath{\geqslant}}
\renewcommand{\geq}{\ensuremath{\geqslant}}
\renewcommand{\epsilon}{\ensuremath{\varepsilon}}
\renewcommand{\phi}{\ensuremath{\varphi}}

\counterwithin{figure}{section}
\counterwithin{equation}{section}
\counterwithin{table}{section}
\newcommand{\sign}[1][5cm]{\makebox[#1]{\hrulefill}} % Поля подписи и даты
\graphicspath{{pics/}} % Путь до папки с картинками
\captionsetup{justification=centering,margin=1cm}
\def\arraystretch{1.3}

\begin{document}

\begin{titlepage}
\begin{center}
	Санкт-Петербургский политехнический университет Петра Великого\\[0.3cm]
	Институт компьютерных наук и технологий \\[0.3cm]
	Кафедра компьютерных систем и программных технологий\\[4cm]
	
	\textbf{Расчётное задание №6}\\[2mm]
	\textbf{Дисциплина:} Системный анализ и принятие решений\\[2mm]
	\textbf{Тема:} Дискретное программирование. Задача коммивояжёра\\[2mm]
	Вариант 39\\[6.5cm]
\end{center}

\begin{flushleft}
	\hspace*{5mm} Выполнил студент гр. 33501/4  \hspace*{3cm}\sign[3cm]\hspace*{2mm} А.Ю. Ламтев\\
	\hspace*{10.85cm} (подпись)\\[2.5mm]
	\hspace*{5mm} Преподаватель \hspace*{6.45cm}\sign[3cm]\hspace*{2mm} С.С. Сабонис\\
	\hspace*{10.85cm} (подпись)\\[2.5mm]
	\hspace*{11.1cm} <<\underline{\the\day}>> \underline{\hspace{5mm}ноября\hspace{5mm}} \the\year\hspace{1mm} г.
\end{flushleft}

\vfill

\begin{center}
	Санкт-Петербург\\
	\the\year
\end{center}
\end{titlepage}
\addtocounter{page}{1}

\section{Задание}

Задана сеть массового обслуживания, включающая $M = 4$ узла и источник с интенсивностью $\lambda_0$, заданы матрица передач и описание узлов как систем массового обслуживания (число каналов, интенсивность обслуживания).

\begin{itemize}
	\item $\lambda_0 = 0.5$
	\item Матрица передач:
	\begin{equation*}
		\mathbb{P} = \begin{pmatrix}
			0 & 1 & 0 & 0 & 0 \\
			\sfrac{1}{9} & 0 & \sfrac{7}{18} & 0 & \sfrac{1}{2} \\
			0 & \sfrac{7}{11} & 0 & 0 & \sfrac{4}{11} \\
			0 & \sfrac{7}{12} & \sfrac{5}{12} & 0 & 0 \\
			0 & 0 & 0 & 1 & 0
		\end{pmatrix}
	\end{equation*}
	\item 1 узел: система $M/M/3$, $\mu = 9$
	\item 2 узел: система $M/M/3$, $\mu = 4$
	\item 3 узел: система $M/M/1$, $\mu = 9$
	\item 4 узел: система $M/M/2$, $\mu = 4$
\end{itemize}

Необходимо:

\begin{enumerate}
	\item Построить граф сети;
	\item Выяснить, может ли данная сеть работать в установившемся режиме;
	\item Для заданной интенсивности $\lambda_0$:
		\begin{enumerate}[label*=\arabic*]
			\item Определить наиболее и наименее нагруженные узлы;
			\item Определить среднее число требований, среднее число ожидающих требований,
			среднее время пребывания и среднее время ожидания для каждого узла;
			\item Определить среднее число требований, среднее число ожидающих требований,
			среднее время пребывания и среднее время ожидания для всей сети;
			\item Результаты оформить в итоговой таблице.
		\end{enumerate}
\end{enumerate}

\section{Построение графа сети}

На рис. \ref{pic:nmo-graph} изображён граф сети.

\begin{figure}[H]
\begin{center}
\begin{tikzpicture}[->,>=stealth',shorten >=2pt,auto,node distance=5cm,
                    thick,main node/.style={circle,draw,font=\sffamily\Large\bfseries}]

  \node[main node] (0) {0};
  \node[main node] (1) [right of = 0]{1};
  \node[main node] (2) [above right of = 1] {2};
  \node[main node] (3) [below right of = 2] {3};
  \node[main node] (4) [below left of = 3] {4};

  \path[every node/.style={font=\sffamily\small}]
    (0) edge [bend left] node[above] {$1$} (1)
    (1) edge [bend left] node[above] {$\sfrac{7}{18}$} (2)
    (1) edge [bend right] node[below] {$\sfrac{1}{2}$} (4)
    (2) edge [bend left] node[below] {$\sfrac{7}{11}$} (1)
    (2) edge             node[pos=0.8] {$\sfrac{4}{11}$} (4)
    (3) edge [bend right] node[above] {$\sfrac{5}{12}$} (2)
    (3) edge             node[pos=0.8] {$\sfrac{7}{12}$} (1)
    (4) edge [bend right] node[below] {$1$} (3)
    
    (1) edge [bend left] node[below] {$\sfrac{1}{9}$} (0);
  
\end{tikzpicture}
\caption{Граф сети}
\label{pic:nmo-graph}
\end{center}
\end{figure}

\section{Вычисление интенсивностей потоков в узлах}

Вычислим коэффициенты передачи от источника к $i$-ому узлу $\alpha_i = \sfrac{\lambda_i}{\lambda_0}$:
\begin{equation*}
	\begin{cases}
		\lambda_0 = \sfrac{1}{9} \lambda_1 \\
		\lambda_1 = \lambda_0 + \sfrac{7}{11} \lambda_2 + \sfrac{7}{12} \lambda_3 \\
		\lambda_2 = \sfrac{7}{18} \lambda_1 + \sfrac{5}{12} \lambda_3 \\
		\lambda_3 = \lambda_4 \\
		\lambda_4 = \sfrac{1}{1} \lambda_1 + \sfrac{4}{11} \lambda_2 \\
	\end{cases} \Leftrightarrow \begin{cases}
		\lambda_1 = 9 \lambda_0 \\
		\lambda_2 = \sfrac{4257}{672} \lambda_0 \\
		\lambda_3 = \sfrac{381}{56} \lambda_0 \\
		\lambda_4 = \sfrac{381}{56} \lambda_0
	\end{cases} \Rightarrow \begin{cases}
		\alpha_1 = 9 \\
		\alpha_2 = \sfrac{4257}{672} \\
		\alpha_3 = \sfrac{381}{56} \\
		\alpha_4 = \sfrac{381}{56}
	\end{cases}
\end{equation*}

\section{Проверка наличия установившегося режима}

Выполним проверку условия наличия установившегося режима сети:

\begin{equation*}
	\lambda_0 < \min \limits_{(j)} \left( \frac{m_j \cdot \mu_j}{\alpha_j} \right)
\end{equation*}

\begin{equation*}
	 \min \limits_{(j)} \left( \frac{m_j \cdot \mu_j}{\alpha_j} \right) \approx min \left(3,\ 1.9,\ 1.3,\ 1.2 \right) = 1.2 > \lambda_0 = 0.5
\end{equation*}

Режим установился.

\section{Разрезание сети}

Выполним разрезание сети и рассчитаем локальные показатели узлов сети, после чего произведем <<сборку>> сети, рассчитав интегральные показатели сети по теореме Джонсона. В таблице \ref{tab:results} приведены результаты рассчитанных показателей для отдельных узлов и для всей сети.
\begin{table}[H]
	\begin{center}
		\caption{Результаты}
		\label{tab:results}
		\def\tabcolsep{4pt}
		\begin{tabular}{|c|c|c|c|c|c|}
			\hline
			& 1 узел & 2 узел & 3 узел & 4 узел & Вся сеть \\
			\hline
			Ср. число требований $\not{n}$ & $0.503 $ & $0.81  $ & $0.6077$ & $1.0382$ & $2.9589$ \\
			\hline
			Ср. число ожидающих требований $\not{l}$ & $0.003 $ & $0.0182$ & $0.2297$ & $0.1877$ & $0.4386$ \\
			\hline
			Ср. время пребывания $\not{t_\text{с}}$ & $0.1118$ & $0.2557$ & $0.1786$ & $0.3052$ & $5.9178$ \\
			\hline
			Ср. время ожидания $\not{t_\text{ож}}$ & $0.0007$ & $0.0057$ & $0.0675$ & $0.0552$ & $0.8772$ \\
			\hline	
		\end{tabular}
	\end{center}
\end{table}

\end{document}