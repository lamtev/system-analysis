\documentclass[a4paper,14pt]{extarticle}

\usepackage[utf8x]{inputenc}
\usepackage[T1,T2A]{fontenc}
\usepackage[russian]{babel}
\usepackage{hyperref}
\usepackage{indentfirst}
\usepackage{here}
\usepackage{array}
\usepackage[table]{xcolor}
\usepackage{datetime}
\usepackage{multirow}
\usepackage{hhline}
\usepackage{mathtools,cancel}
\usepackage{forest}
\usepackage{graphicx}
\usepackage{caption}
\usepackage{subcaption}
\usepackage{chngcntr}
\usepackage{amsmath}
\usepackage{amssymb}
\usepackage{pgfplots}
\usepackage{pgfplotstable}
\usepackage[left=2cm,right=2cm,top=2cm,bottom=2cm,bindingoffset=0cm]{geometry}
\usepackage{multicol}
\usepackage{askmaps}
\usepackage{tikz}

\newcommand*\circled[1]{\tikz[baseline=(char.base)]{
            \node[shape=circle,draw,inner sep=2pt] (char) {#1};}}

\DeclareMathOperator*{\argmin}{argmin}

\renewcommand{\not}[1]{\mkern 1.5mu\overline{\mkern-1.5mu#1\mkern-1.5mu}\mkern 1.5mu}
\renewcommand{\le}{\ensuremath{\leqslant}}
\renewcommand{\leq}{\ensuremath{\leqslant}}
\renewcommand{\ge}{\ensuremath{\geqslant}}
\renewcommand{\geq}{\ensuremath{\geqslant}}
\renewcommand{\epsilon}{\ensuremath{\varepsilon}}
\renewcommand{\phi}{\ensuremath{\varphi}}

\counterwithin{figure}{section}
\counterwithin{equation}{section}
\counterwithin{table}{section}
\newcommand{\sign}[1][5cm]{\makebox[#1]{\hrulefill}} % Поля подписи и даты
\graphicspath{{pics/}} % Путь до папки с картинками
\captionsetup{justification=centering,margin=1cm}
\def\arraystretch{1.3}

\begin{document}

\begin{titlepage}
\begin{center}
	Санкт-Петербургский политехнический университет Петра Великого\\[0.3cm]
	Институт компьютерных наук и технологий \\[0.3cm]
	Кафедра компьютерных систем и программных технологий\\[4cm]
	
	\textbf{Расчётное задание №6}\\[2mm]
	\textbf{Дисциплина:} Системный анализ и принятие решений\\[2mm]
	\textbf{Тема:} Дискретное программирование. Задача коммивояжёра\\[2mm]
	Вариант 39\\[6.5cm]
\end{center}

\begin{flushleft}
	\hspace*{5mm} Выполнил студент гр. 33501/4  \hspace*{3cm}\sign[3cm]\hspace*{2mm} А.Ю. Ламтев\\
	\hspace*{10.85cm} (подпись)\\[2.5mm]
	\hspace*{5mm} Преподаватель \hspace*{6.45cm}\sign[3cm]\hspace*{2mm} С.С. Сабонис\\
	\hspace*{10.85cm} (подпись)\\[2.5mm]
	\hspace*{11.1cm} <<\underline{\the\day}>> \underline{\hspace{5mm}ноября\hspace{5mm}} \the\year\hspace{1mm} г.
\end{flushleft}

\vfill

\begin{center}
	Санкт-Петербург\\
	\the\year
\end{center}
\end{titlepage}
\addtocounter{page}{1}

\section{Задание}

Дана задача нелинейного программирования:

\begin{equation}
\label{eq:target}
	max \left( -17 x^2_1 - 23 x^2_2 + 8 x_1 x_2 + 182 x_1 + 266 x_2 \right)
\end{equation}

\begin{enumerate}

	\item Решить задачу методом Лагранжа при ограничении: 
	
		\begin{equation}
		\label{eq:lagrange-constraints}
			0 \cdot x_1 + 1 \cdot x_2 = 4
		\end{equation}
	
	\item Записать необходимые условия оптимальности для задачи при ограничениях:
	
		\begin{equation}
		\label{eq:bil-constraints}
			\begin{cases}
				9 x_1 + 7 x_2 \leq 63
				\\
				-x_1 + x_2 \leq 4
				\\
				-x_1 \leq 0
				\\
				-x_2 \leq 0
			\end{cases}
		\end{equation}
	
	\item Решить задачу методом наискорейшего подъема.
	
	\item Решить задачу методом Ньютона.	
	
	\item Решить задачу методом сопряжённых градиентов.

	\item Решить задачу методом Бройдена.

\end{enumerate}

\section{Решение}

\subsection{Метод Лагранжа}

Запишем функцию Лагранжа для целевой функции \ref{eq:target} и заданного ограничения \ref{eq:lagrange-constraints}

\begin{equation*}
	L(X, V) = -17 x^2_1 - 23 x^2_2 + 8 x_1 x_2 + 182 x_1 + 266 x_2 + v(x_2 - 4)
\end{equation*}

Запишем условие стационарности $L(X, V)$ в точке $(X^*, V^*)$:

\begin{equation*}
	\begin{cases}
		\frac{\partial L}{\partial x_1} = -34 x_1 + 8 x_2 + 182 = 0
		\\
		\frac{\partial L}{\partial x_2} = -46 x_2 + 8 x_1 + 266 + v = 0
		\\
		\frac{\partial L}{\partial v} = x_2 - 4 = 0
	\end{cases}
\end{equation*}

Найдём $(X^*,\ V^*)$:

\begin{equation*}
	(X^*,\ V^*) = \left( \frac{17}{107},\ 4,\ -\frac{7840}{107} \right)
\end{equation*}

Вычислим матрицу Гессе для функции $L(X, V)$:

\begin{equation*}
	H = \begin{pmatrix}
		-34 & 8
		\\
		8 & -46
	\end{pmatrix}
\end{equation*}

Согласно критерию Сильвестра $H$ - отрицательно определённая. Следовательно $(X^*,\ V^*)$ -- точка максимума $L(X, V)$, а точка $X^* = \left(\frac{17}{107},\ 4 \right)$ -- решение поставленной задачи.

\subsection{Условия оптимальности}

Запишем необходимые условия оптимальности для задачи с ограничениями. Преобразуем систему с ограничениями \ref{eq:bil-constraints} к следующему виду:

\begin{equation*}
	\begin{cases}
		9 x_1 + 7 x_2 - 63 \leq 0
		\\
		-x_1 + x_2 - 4 \leq 0
		\\
		x_1,\ x_2 \geq 0
	\end{cases}
\end{equation*}

В данном случае необходимым условием оптимальности будет следующее условие Куна-Такера:

\begin{equation}
\label{eq:kt}
	\begin{cases}
		f^{'}(X^*) + \sum\limits_{j = 1}^J u_j \cdot g^{'}_j(X^*) = 0
		\\
		u_j \cdot g_j(X^*) = 0,\ j = 1 \dots J
		\\
		u_j \leq 0,\ \text{если}\ g_j(X) \leq 0
		\\
		u_j \geq 0,\ \text{если}\ g_j(X) \geq 0
	\end{cases}
\end{equation}

Вычислим градиенты целевой функции и функций-ограничений:

\begin{equation*}
	\nabla f(X) = \left(  \frac{\partial f}{\partial x_1}(X) \hspace{7mm} \frac{\partial f}{\partial x_2}(X) \right)^T = \begin{pmatrix}
		-34 x_1 + 8 x_2 + 182
		\\
		-46 x_2 + 8 x_1 + 266
	\end{pmatrix}
\end{equation*}

\begin{equation*}
	\nabla g_1 = \left( 9 \hspace{5mm} 7 \right)^T
\end{equation*}

\begin{equation*}
	\nabla g_2 = \left( -1 \hspace{5mm} 1 \right)^T
\end{equation*}

\begin{equation*}
	\nabla g_3 = \left( 1 \hspace{5mm} 0 \right)^T
\end{equation*}

\begin{equation*}
	\nabla g_4 = \left( 0 \hspace{5mm} 1 \right)^T
\end{equation*}

Подставим вычисленные градиенты в уравнение \ref{eq:kt} и получим необходимые условия оптимальности $X^*$:

\begin{equation*}
	\begin{cases}
		\left( -34 x_1 + 8 x_2 + 182 \right) \Big|_{(x_1, x_2) = X^*} + u_1 \cdot 9 + u_2 \cdot (-1) + u_3 \cdot 1 + u_4 \cdot 0 = 0
		\\
		\left( -46 x_2 + 8 x_1 + 266 \right) \Big|_{(x_1, x_2) = X^*} + u_1 \cdot 7 + u_2 \cdot 1 + u_3 \cdot 0 + u_4 \cdot 1 = 0
		\\
		u_1 \cdot (9 x_1 + 7 x_2 - 63) \Big|_{(x_1, x_2) = X^*} = 0
		\\
		u_2 \cdot (-x_1 + x_2 - 4) \Big|_{(x_1, x_2) = X^*} = 0
		\\
		u_3 \cdot x_1 \Big|_{(x_1, x_2) = X^*} = 0
		\\
		u_4 \cdot x_2 \Big|_{(x_1, x_2) = X^*} = 0
		\\
		u_1 \leq 0
		\\
		u_2 \leq 0
		\\
		u_3 \geq 0
		\\
		u_4 \geq 0\
	\end{cases}
\end{equation*}

\subsection{Метод Била}

Введём дополнительные переменные $x_3$ и $x_4$ и преобразуем систему ограничений \ref{eq:bil-constraints} к следующему виду:

\begin{equation*}
\begin{cases}
	9 x_1 + 7 x_2 + x_3 = 63
	\\
	- x_1 + x_2 + x_4 = 4
	\\
	x_i \geq 0,\ i = 1 \dots 4
\end{cases}
\end{equation*}

\paragraph{Шаг 1. Формирование начального базиса $\text{Б}_0$}

Пусть $\text{Б}_0 = (3,\ 4)$, тогда $X^{(0)} = (0,\ 0,\ 63,\ 4)^T$ -- допустимое базисное решение.

\paragraph{Шаг 2. Построение и анализ симплекс-таблицы.}

Построим симплекс-таблицу \ref{tab:simpl:1}:

\begin{table}[H]
\begin{center}
	\caption{Симплекс-таблица для базисного решения $X^{(0)}$}
	\label{tab:simpl:1}
	\def\tabcolsep{18pt}
	\def\arraystretch{1.5}
	\fontsize{13}{14}\selectfont
	\begin{tabular}{|c|c||c||c|}
		\hline 
		$X^{(0)}$ & $x_1$ & $x_2$ & $b$ \\ 
		\hline 
		$x_3$ & -9 & -7 & 63 \\ 
		\hhline{|=|=||=|=|}
		$x_4$ & 1 & \cellcolor{pink} -1 & 4 \\ 
		\hhline{|=|=||=|=|}
		$\frac{\partial f}{\partial x_r} \left(X^{(0)} \right)$ & 182 & 266 &  \\ 
		\hline 
	\end{tabular} 
\end{center}
\end{table}

Т.к. $c = (182\ \ 266)^T > 0$, базисное решение $X^{(0)}$ -- неоптимально.

Разрезающий столбец --- $x_2$, т.к. ей соответствует максимальная производная целевой функции --- 266.

Определим разрезающую строку. Для этого найдём соотношение между приращением свободной переменной $x_2$ и изменениями базисных переменных $x_3$, $x_4$ и производной $\frac{\partial f}{\partial x_2} \left(X \right)$:

\begin{equation*}
	\frac{\partial f}{\partial x_2} \left(X \right) = 0 \Rightarrow x_2 = \frac{266}{46} \approx 5.78
\end{equation*}

\begin{equation*}
	x_3 = 0\ \ \Rightarrow x_2 = \frac{63}{7} = 9
\end{equation*}

\begin{equation*}
	x_4 = 0\ \ \Rightarrow x_2 = \frac{4}{1} = 4
\end{equation*}

Переменная $x_4$ обращается в нуль раньше остальных, поэтому ей соответствует разрезающая строка.

Разрезающий элемент $-1$ выделен цветом в симплекс-таблице \ref{tab:simpl:1}. 

\paragraph{Шаг 3. Пересчёт симплекс-таблицы для базисного решения $X^{(1)}$.}

Пересчитаем симплекс-таблицу \ref{tab:simpl:1} и получим симплекс-таблицу \ref{tab:simpl:2}.

\begin{table}[H]
\begin{center}
	\caption{Симплекс-таблица для базисного решения $X^{(1)}$}
	\label{tab:simpl:2}
	\def\tabcolsep{18pt}
	\def\arraystretch{1.5}
	\fontsize{13}{14}\selectfont
	\begin{tabular}{|c||c||c|c|}
		\hline 
		$X^{(0)}$ & $x_1$ & $x_4$ & $b$ \\ 
		\hhline{|=||=||=|=|} 
		$x_3$ & \cellcolor{pink} -16 & 7 & 35 \\ 
		\hhline{|=||=||=|=|}
		$x_2$ & 1 & -1 & 4 \\ 
		\hline
		$\frac{\partial f}{\partial x_r} \left(X^{(0)} \right)$ & 448 & -266 &  \\ 
		\hline 
	\end{tabular} 
\end{center}
\end{table}

\begin{equation*}
	f(X) = -17 x_1^2 - 23 (x_1 - x_4)^2 + 8 x_1 (x_1 - x_4) + 182 x_1 + 266 (x_1 - x_4)
\end{equation*}

Решение $X^{(1)} = (0,\ 4,\, 35,\ 0)^T$ --- допустимое, но не оптимальное, т.к. $c = (448\ \ -266)^T \nleq 0$.

Разрезающий столбец --- $x_1$, т.к. ей соответствует максимальная производная целевой функции --- 448.

Определим разрезающую строку. Для этого найдём соотношение между приращением свободной переменной $x_1$ и изменениями базисных переменных $x_2$, $x_3$ и производной $\frac{\partial f}{\partial x_1} \left(X \right)$:

\begin{equation*}
	\frac{\partial f(X)}{\partial x_1} = -64 x_1 + 38 x_4 + 448
\end{equation*}

\begin{equation*}
	\frac{\partial f(X)}{\partial x_1} = 0 \Rightarrow x_1 = \frac{448}{64} = 7
\end{equation*}

\begin{equation*}
	x_3 = 0\ \ \Rightarrow x_1 = \frac{35}{16} \approx 2.188
\end{equation*}

\begin{equation*}
	x_2 = 0 \Rightarrow \ \ \ x_1 = \frac{4}{1} = 4
\end{equation*}

Переменная $x_3$ обращается в нуль раньше остальных, поэтому ей соответствует разрезающая строка.

Разрезающий элемент $-16$ выделен цветом в симплекс-таблице \ref{tab:simpl:2}. 

\paragraph{Шаг 4. Пересчёт симплекс-таблицы базисного для решения $X^{(2)}$.}

Пересчитаем симплекс-таблицу \ref{tab:simpl:2} и получим симплекс-таблицу \ref{tab:simpl:3}.

\begin{table}[H]
\begin{center}
	\caption{Симплекс-таблица для базисного решения $X^{(2)}$}
	\label{tab:simpl:3}
	\def\tabcolsep{18pt}
	\def\arraystretch{1.5}
	\fontsize{13}{14}\selectfont
	\begin{tabular}{|c|c|c|c|}
		\hline 
		$X^{(0)}$ & $x_3$ & $x_4$ & $b$ \\ 
		\hline
		$x_1$ & $-\frac{1}{16}$ & $\frac{7}{16}$ & \cellcolor{green} $\frac{35}{16}$ \\ 
		\hline
		$x_2$ & $-\frac{1}{16}$ & -$\frac{9}{16}$ & \cellcolor{green} $\frac{99}{16}$ \\ 
		\hline
		$\frac{\partial f}{\partial x_r} \left(X^{(0)} \right)$ & -28 & -70 &  \\ 
		\hline 
	\end{tabular} 
\end{center}
\end{table}


Решение $X^{(2)} = \left(\frac{35}{16},\ \frac{99}{16} ,\, 0,\ 0\right)^T$ --- допустимое и оптимальное, т.к. $c = (-28\ \ -70)^T < 0$.

\begin{equation*}
	X^* = \left(\frac{35}{16},\ \frac{99}{16} \right)\ \text{-- решение исходной задачи условной оптимизиции.}
\end{equation*}

\subsection{Метод проекции градиента}

Запишем ограничения \ref{eq:bil-constraints} в матричной форме:

\begin{equation*}
	A \cdot X \leq b, 
\end{equation*}

\begin{equation*}
	A = \begin{pmatrix}
		9 & 7
		\\
		-1 & 1
		\\
		-1 & 0
		\\
		0 & -1
	\end{pmatrix}
	,\ %
	X = \begin{pmatrix}
	x_1 \\ x_2
	\end{pmatrix}
	,\ %
	b = \begin{pmatrix}
		64 \\ 4 \\ 0 \\ 0
	\end{pmatrix}
\end{equation*}

Вычислим градиент и гессиан целевой функции:

\begin{equation*}
	\nabla f(X) = \begin{pmatrix}
		-34 x_1 + 8 x_2 + 182
		\\
		-46 x_2 + 8 x_1 + 266
	\end{pmatrix}%
	,\ \ %
	H = \begin{pmatrix}
		-34 & 8
		\\
		8 & -46	
	\end{pmatrix}
\end{equation*}

В качестве начальной точки выберем $X^{(0)} = (0,\ 0)^T$

\paragraph{Шаг 1. Определение матрицы активных ограничений в начальной точке.}

Убедимся в допустимости градиентного направления:

\begin{equation*}
	A \nabla f(X^{(0)}) = \begin{pmatrix}
	-1 && 0 \\ 0 && -1
	\end{pmatrix}
	\begin{pmatrix}
	182 \\ 266
	\end{pmatrix}
	=
	\begin{pmatrix}
	-182 \\ -266
	\end{pmatrix}
	< 0,
\end{equation*}

поэтому $K^{(0)} = \nabla f(X^{(0)}) = \begin{pmatrix}
	182 & 266
	\end{pmatrix}^T$

\paragraph{Шаг 2. Выбор длины шага $t^{(0)}$}

Найдём множество $I_{prev}$ индексов нарушаемых ограничений.

\begin{equation*}
	A K^{(0)} = \begin{pmatrix}
		9 & 7
		\\
		-1 & 1
		\\
		-1 & 0
		\\
		0 & -1
	\end{pmatrix}
	\begin{pmatrix}
		182 \\ 266
	\end{pmatrix}
	=
	\begin{pmatrix}
		3500 \\ 84 \\ -182 \\ -266
	\end{pmatrix}
	\Rightarrow I_{prev} = \{1,\ 2\}
\end{equation*}

Тогда $t^{(0)} = min\{t^*,\ t_{prev_1},\ t_{prev_2}\}$

\begin{equation*}
	t^* = -\frac{\nabla^T f(X^{(0)}) K^{(0)}}{(K^{(0)})^T H K^{(0)}} = -\frac{(182\ 266) \begin{pmatrix} 182 \\ 266 \end{pmatrix}}{(182\ 266) \begin{pmatrix} -34 & 8 \\ 8 & -46 \end{pmatrix} \begin{pmatrix} 182 \\ 266 \end{pmatrix}} = 0.0288
\end{equation*}

\begin{equation*}
	t_{prev_1} = \frac{b_1 - a_1 X^{(0)}}{a_1 K^{(0)}} = \frac{64 - (9\ 7) \cdot 0}{(9\ 7) \begin{pmatrix} 182 \\ 266 \end{pmatrix}} = \frac{64}{3500} = 0.0183
\end{equation*}

\begin{equation*}
	t_{prev_2} = \frac{b_2 - a_2 X^{(0)}}{a_2 K^{(0)}} = \frac{4 - (-1\ 1) \cdot 0}{(-1\ 1) \begin{pmatrix} 182 \\ 266 \end{pmatrix}} = \frac{4}{84} = 0.0476
\end{equation*}

\begin{equation*}
	t^{(0)} = t_{prev_1} = 0.0183
\end{equation*}

\begin{equation*}
	X^{(1)} = X^{(0)} + t^{(0)} K^{(0)} = \begin{pmatrix}
		0 \\ 0 
	\end{pmatrix}
	+
	0.0183 \cdot \begin{pmatrix}
		182 \\ 266 
	\end{pmatrix}
	=
	\begin{pmatrix}
		3.3306 \\ 4.8678 
	\end{pmatrix}
\end{equation*}

\paragraph{Шаг 3. Формирование матрицы активных ограничений}

Определим оператор проекции и направлние $K^{(1)}$:

\begin{equation*}
	A = \begin{pmatrix} 9 & 7 \end{pmatrix},\ %
	\nabla f(X^{(1)}) = \begin{pmatrix} 107.702 & 68.726 \end{pmatrix}^T
\end{equation*}

\begin{equation*}
	A \nabla f(X^{(0)}) = \begin{pmatrix} 9 & 7 \end{pmatrix}
	\begin{pmatrix} 107.702 \\ 68.726 \end{pmatrix}
	=
	1450.4
	> 0,
\end{equation*}

направление градиента недопустимо.

\begin{equation*}
P = E - A^T \left(A A^T\right)^{-1} A = \begin{pmatrix} 1 & 0 \\ 0 & 1 \end{pmatrix} - \begin{pmatrix} 0.6231 & 0.4846 \\ 0.4846 & 0.3769 \end{pmatrix} = \begin{pmatrix} 0.3769 & -0.4846 \\ -0.4846 & 0.6231 \end{pmatrix}
\end{equation*}

\begin{equation*}
	K^{(1)} = P \nabla f(X^{(1)}) = \begin{pmatrix} 0.3769 & -0.4846 \\ -0.4846 & 0.6231 \end{pmatrix} \begin{pmatrix} 107.702 \\ 68.726 \end{pmatrix} = \begin{pmatrix} 7.288 \\ -9.369 \end{pmatrix}
\end{equation*}

\paragraph{Шаг 4. Выбор длины шага $t^{(1)}$}

Найдём множество $I_{prev}$ индексов нарушаемых ограничений.

\begin{equation*}
	A K^{(1)} = \begin{pmatrix}
		9 & 7
		\\
		-1 & 1
		\\
		-1 & 0
		\\
		0 & -1
	\end{pmatrix}
	\begin{pmatrix} 7.288 \\ -9.369 \end{pmatrix}
	=
	\begin{pmatrix}
		0 \\ -16.657 \\ -7.288 \\ 9.369
	\end{pmatrix}
	\Rightarrow I_{prev} = \{4\}
\end{equation*}

Тогда $t^{(0)} = min\{t^*,\ t_{prev_4}\}$

\begin{equation*}
	t^* = -\frac{\nabla^T f(X^{(1)}) K^{(1)}}{(K^{(1)})^T H K^{(1)}} = -\frac{(107.702\  68.726) \begin{pmatrix} 7.288 \\ -9.369 \end{pmatrix}}{(7.288\ -9.369) \begin{pmatrix} -34 & 8 \\ 8 & -46 \end{pmatrix} \begin{pmatrix} 7.288 \\ -9.369 \end{pmatrix}} = 0.02033
\end{equation*}

\begin{equation*}
	t_{prev_4} = \frac{b_4 - a_4 X^{(1)}}{a_4 K^{(1)}} = \frac{0 - (0\ -1) \cdot \begin{pmatrix} 3.3306 \\ 4.8678 \end{pmatrix}}{(0\ -1) \begin{pmatrix} 7.288 \\ -9.369 \end{pmatrix}} = \frac{4.8678}{9.369} = 0.519
\end{equation*}

\begin{equation*}
	t^{(0)} = t^* = 0.02033
\end{equation*}

\begin{equation*}
	X^{(2)} = X^{(1)} + t^{(1)} K^{(1)} = \begin{pmatrix}
		3.3306 \\ 4.8678
	\end{pmatrix}
	+
	0.02033 \cdot \begin{pmatrix} 7.288 \\ -9.369 \end{pmatrix}
	=
	\begin{pmatrix}
		3.479 \\ 4.677
	\end{pmatrix}
\end{equation*}

\paragraph{Шаг 5. Выбор направления}



\begin{equation*}
	A = (0\ -1),\ 
	\nabla f(X^{(2)}) = \begin{pmatrix} 
		101.2544
		\\ 
		78.7437
	\end{pmatrix}
\end{equation*}

\begin{equation*}
	K^{(2)} = P \nabla f(X^{(2)}) = \begin{pmatrix} 
		0.3769 & -0.4845 
		\\ 
		-0.4845 & 0.6231 
	\end{pmatrix}
	\begin{pmatrix} 
		101.2544
		\\ 
		78.7437
	\end{pmatrix}
	=
	\begin{pmatrix} 
		0.0115
		\\ 
		0.0074
	\end{pmatrix}
\end{equation*}

\end{document}